%%%% ximera preamble extract



%% TikZ/PGFplot related

\usepackage{tkz-euclide}
\usepackage{tikz}
\usepackage{tikz-cd}
\usepackage{pgffor} %% required for integral for loops
\usetikzlibrary{arrows}
\tikzset{>=stealth,commutative diagrams/.cd,
  arrow style=tikz,diagrams={>=stealth}} %% cool arrow head
\tikzset{shorten <>/.style={ shorten >=#1, shorten <=#1 } } %% allows shorter vectors

\usetikzlibrary{backgrounds} %% for boxes around graphs
\usetikzlibrary{shapes,positioning}  %% Clouds and stars
\usetikzlibrary{matrix} %% for matrix
\usepgfplotslibrary{polar} %% for polar plots
\usetkzobj{all}



%% colors

\colorlet{textColor}{black}
\colorlet{background}{white}
\colorlet{penColor}{blue!50!black} % Color of a curve in a plot
\colorlet{penColor2}{red!50!black}% Color of a curve in a plot
\colorlet{penColor3}{red!50!blue} % Color of a curve in a plot
\colorlet{penColor4}{green!50!black} % Color of a curve in a plot
\colorlet{penColor5}{orange!80!black} % Color of a curve in a plot
\colorlet{penColor6}{yellow!70!black} % Color of a curve in a plot
\colorlet{fill1}{penColor!20} % Color of fill in a plot
\colorlet{fill2}{penColor2!20} % Color of fill in a plot
\colorlet{fillp}{fill1} % Color of positive area
\colorlet{filln}{penColor2!20} % Color of negative area
\colorlet{fill3}{penColor3!20} % Fill
\colorlet{fill4}{penColor4!20} % Fill
\colorlet{fill5}{penColor5!20} % Fill
\colorlet{gridColor}{gray!50} % Color of grid in a plot
\newcommand{\surfaceColor}{violet}
\newcommand{\surfaceColorTwo}{redyellow}
\newcommand{\sliceColor}{greenyellow}



%% image environment

\NewEnviron{image}[1][3in]{%
  \begin{center}\resizebox{#1}{!}{\BODY}\end{center}% resize and center
}



%% more packages

\usepackage[makeroom]{cancel} %% for strike outs
\usepackage{multicol}
\usepackage{array}



%% lengths

\setlength{\extrarowheight}{+.1cm}


%% macros
% \newcommand{\dd}[2][]{\frac{\d #1}{\d #2}} % conflict
\newcommand{\pp}[2][]{\frac{\partial #1}{\partial #2}}
\newcommand{\ddx}{\frac{d}{\d x}}
\newcommand{\dfn}{\textbf}
\newcommand{\unit}{\mathop{}\!\mathrm}
\newcommand{\eval}[1]{\bigg[ #1 \bigg]}
\newcommand{\seq}[1]{\left( #1 \right)}
\renewcommand{\d}{\mathop{}\!d}
\renewcommand{\l}{\ell}
